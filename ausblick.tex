\section{Ausblick}
\begin{frame}{}
    \begin{center}
        Ausblick
     \end{center}
\end{frame}

\begin{frame}{Aktuelle Probleme}
    \begin{itemize}
        \item KeyXchange
        \begin{itemize}
            \item VPN Schlüsseltausch über zentrales Tool
            \begin{itemize}
                \item Manuelle Zuweisung der VPN Server
                \item Keine Anpassung an Firmware
                \item[$\rightarrow$] Loop durch versehentliches Meshing
            \end{itemize}
        \end{itemize}
        \item L3-Netz
        \begin{itemize}
            \item Kein Traffic Balancing zwischen VPN Server und Gateways
            \item Keine Möglichkeit die L2 Netze per Richtfunk zu verbinden
        \end{itemize}
        \item Monitoring:
        \begin{itemize}
            \item Ein Alfred-Proxy für alle Hoods
        \end{itemize}
        \item Firmware-Build:
        \begin{itemize}
            \item OpenWrt Packages
        \end{itemize}
    \end{itemize}
\end{frame}

\begin{frame}{KeyXchange}
    \begin{itemize}
        \item Dezentralisieren: Knoten soll selber seine Hood finden
        \begin{itemize}
            \item VPN Server auswählen
            \item VPN Server alle Verbindungen akzeptieren
        \end{itemize}
        \item WiFi Settings auswählen
        \begin{itemize}
            \item Problem: wenn z.B. kein VPN da ist?
        \end{itemize}
        \item Zwei Dinge werden benötigt: 
        \begin{itemize}
            \item Mögliche VPN Server
            \item Daten der Hood (WiFi / Routing / etc)
        \end{itemize}
    \end{itemize}
\end{frame}

\begin{frame}{Hood-Configs}
    \begin{columns}
    \column{0.45\textwidth}
        \begin{block}{Hood Config}
            \footnotesize
            %[general]\\
            %version=1\\[1ex]
            [hood]\\
            name=fuerth\\
            bssid=ca:ff:ee:ba:be:00\\
            protocol=batman-adv-v14\\
            channel2=1\\
            mode2=ht20+\\
            type=802.11s\\
            location=49.4814;10.966\\
            noAutoConnect=true\\[1ex]
            [network]\\
            prefix=fdff:1::
        \end{block}
    \column{0.45\textwidth}
        \begin{block}{Gateway Config}
            \footnotesize
            %[general]\\
            %version=1\\[1ex]
            [network]\\
            hood=fuerth\\[1ex]
            %ip4range=10.50.44.1-.46.255\\
            %//ip6range=2001:xx:xx:xx/64\\[1ex]
            [vpn]\\
            protocol=fastd\\
            address=vpn1.fff.community\\
            port=10000\\
            %key=$<$fastd-public-key$>$\\[1ex]
        \end{block}
        \begin{block}{Key File}
            \footnotesize
            [sign]\\
            key=$<$public-key$>$\\
        \end{block}
    \end{columns}
\end{frame}

\begin{frame}{Hood-Configs verteilen}
    \begin{itemize}
        \item Download über HTTP Server
            \begin{itemize}
                \item Immer noch zentral
                \begin{itemize}
                    \item[$\rightarrow$] .. später mal sync daemon?
                \end{itemize}
            \end{itemize}
        \item Überschreiben/Löschen absichern
        \begin{itemize}
            \item[$\rightarrow$] Synchronisation nur mit Signatur
        \end{itemize}
        \item Knoten ohne WAN
        \begin{itemize}
            \item Config Access-Point
        \end{itemize}
    \end{itemize}
\end{frame}

\begin{frame}{L3 Richtfunk}
    \center
    \includegraphics[width=0.9\textwidth]{img/dia/l3richtfunk.pdf}
\end{frame}

\begin{frame}{L3 Richtfunk}
    \begin{itemize}
        \item Batman Netz = Normales Ethernet
        \item OLSR gegen Unfug absichern $\rightarrow$ sOLSR (??)
        \item Problem: Nur ein default Gateway pro Client
        \begin{itemize}
            \item[$\rightarrow$] Das Gateway vom Dach oder das hinter dem DSL?
        \end{itemize}
    \end{itemize}
\end{frame}
