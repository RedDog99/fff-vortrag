\section{Freifunk}
\begin{frame}{}
    \begin{center}
        Freifunk
     \end{center}
\end{frame}

\begin{frame}{Freifunk}
    \begin{itemize}
        \item Nicht nur offenes Netzwerk (WLAN)
        \item Nicht-kommerzielle Initiative für freie (Funk-)Netzwerke
        \begin{itemize}
            \item[$\rightarrow$] Bürger investieren in Eigenregie Zeit, Geld und Enthusiasmus
        \end{itemize}
    \end{itemize}
        \vfill
    \begin{itemize}
        \item Mehr als \glqq{}kostenloses Internet\grqq:
        \begin{itemize}
            \item Unabhängiges / dezentrales Netz
            \item Internet Technologie beim Bürger
            \item[$\rightarrow$] Freies Netzwerken
        \end{itemize}
    \end{itemize}
\end{frame}

\begin{frame}{Das braucht man}
    Das Netz benutzen
    \begin{itemize}
        \item Handelsübliches Wlan Gerät
    \end{itemize}

    \vfill

    Das Netz mit aufbauen
    \begin{itemize}
        \item Hardware (ab ca. 20€)
        \item Ein Partner für das Peering (Nachbar, VPN, etc)
        \begin{itemize}
            \item[$\rightarrow$] vereinfacht in der Freifunk-Firmware
        \end{itemize}
        \item Die Zustimmung zum \glqq{}Pico-Peering Agreement\grqq
    \end{itemize}
\end{frame}

\begin{frame}{Pico-Peering Agreement}
    \only<1>{
        Regelwerk, über grundsätzliche Eigenschaften des Freifunks

        \begin{enumerate}
            \item Freier Transit
            \item Offene Kommunikation
            \item Keine Garantie (Haftungsausschluss)
            \item Nutzungsbestimmungen
            \item Lokale (individuelle) Zusätze
        \end{enumerate}
    }
%    \only<2>{
%        \begin{block}{1. Freier Transit}
%            \begin{itemize}
%                \item Freien Transit über die freie Netzwerkinfrastruktur anbieten
%                \item Daten weder stören noch verändern
%            \end{itemize}
%        \end{block}
%    }
%    \only<3>{
%        \begin{block}{2. Offene Kommunikation}
%            \begin{itemize}
%                \item Alle Informationen zum Verbinden zur Infrastruktur veröffentlichen
%                \item Alle Informationen unter freie Lizenz stellen
%                \item Erreichbarkeit: wenigstens eine E-Mail-Adresse bekanntgeben
%            \end{itemize}
%        \end{block}
%    }
%    \only<4>{
%        \begin{block}{3. Keine Garantie (Haftungsausschluss)}
%            \begin{itemize}
%                \item Es gibt keine Garantie für die Verfügbarkeit / Qualität
%                \item Ohne Gewähr, ohne Garantie oder Verpflichtung jedweder Art
%                \item Kann jeder Zeit ohne weitere Erklärung beschränkt oder eingestellt werden
%            \end{itemize}
%        \end{block}
%    }
%    \only<5>{
%        \begin{block}{4. Nutzungsbestimmungen}
%            \begin{itemize}
%                \item Es kann eine Benutzungsrichtlinie formuliert werden
%                \item Kann Informationen über zusätzliche Dienste enthalten
%                \item Darf nicht Punkt 1 bis 3 widersprechen (siehe Punkt 5)
%            \end{itemize}
%        \end{block}
%    }
%    \only<6>{
%        \begin{block}{5. Lokale (individuelle) Zusätze}
%            \begin{itemize}
%                \item Hier können selbst Ergänzungen zur Vertragsvereinbarung vorgenommen werden
%            \end{itemize}
%        \end{block}
%    }
\end{frame}

