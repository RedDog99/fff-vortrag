\section{Freifunk}

\begin{frame}{Einleitung}
    \begin{itemize}
        \item Freifunk Franken ist lokaler Ableger der Freifunk-Bewegung (freifunk.net)
        \item Nicht-kommerzielle Initiative für freie Funknetzwerke\\
        \begin{itemize}
            \item[$\rightarrow$] Bürger investieren in Eigenregie Zeit, Geld und Enthusiasmus
        \end{itemize}
        \item Nicht nur \glqq{}kostenloses Internet\grqq $\Rightarrow$ \glqq{}freies Netzwerken\grqq\\
        \begin{itemize}
            \item Lokal intressante Dienste zur Verfügung stellen (Webcams)
            \item Text, Musik und Filme über das interne Freifunk-Netz übertragen
            \item Über lokale Dienste Chatten oder Telefonieren
        \end{itemize}
    \end{itemize}
\end{frame}

\begin{frame}{Wie es funktioniert}
    \begin{itemize}
        \item Freifunker stellen WLAN-Router für sich selbst und den Datentransfer der anderen Teilnehmer zur Verfügung
        \begin{itemize}
            \item ggf. mit Anschluss an das www (für VPN)
        \end{itemize}
        \item Benachbarte Router verbinden sich und spannen ein sogenanntes Mesh-Netzwerk auf
        \item Nicht benachbarte Router verbinden sich mittels VPN-Tunnel zum Freifunk
        \item Jegliche Verbindung ins www wird hierüber umgeleitet, um Risiken der Störerhaftung zu entgehen
    \end{itemize}
\end{frame}

\begin{frame}{Was braucht man?}
    \begin{itemize}
        \item Ein günstiger, unterstützter Router (ab ca. 17€)
        \item Eine spezielle Firmware
        \item Die Zustimmung zum \glqq{}Pico-Peering Agreement\grqq
        \begin{itemize}
            \item Regelwerk, das grundsätzliche Eigenschaften eines Freifunk-Netzwerkes sichert
            \begin{enumerate}
                \item Freier Transit
                \item Offene Kommunikation
                \item Keine Garantie (Haftungsausschluss)
                \item Nutzungsbestimmungen
                \item Lokale (individuelle) Zusätze
            \end{enumerate}
            \item Die Freifunk Firmware implementiert diese Grundsätze standardmäßig
        \end{itemize}
    \end{itemize}
\end{frame}

