\section*{Was bisher geschah}
\begin{frame}{Was bisher geschah}
    \begin{center}
        Was bisher geschah
    \end{center}
\end{frame}

\begin{frame}{Erste Vorüberlegung}
    \begin{itemize}
        \item Machen wir Bildung?
        \item Wir wollen ehrlich sein.
        \item Wir wollen spätere Probleme vermeiden.
        \item[$\rightarrow$] Ja, aber nicht ausschließlich!
    \end{itemize}
\end{frame}

\begin{frame}{Satzungprüfung Nürnberg}
    \begin{itemize}
        \item Ablehung am 15.06.2015
        \item Finanzamtmitarbeiter wollte sich nicht mit der Materie befassen
        \item[$\rightarrow$] Erstmal wo anders hin...
    \end{itemize}
\end{frame}

\begin{frame}{Satzungprüfung Fürth}
    \begin{itemize}
        \item Erste Anfrage am 31.8.2015
        \item Viele telefonate folgten
        \item Viele Klarstellungen folgten
        \item Abgrenzung zum Internetverein war erfolgreich
        \item Oktober 2015: Fürth hat das Landesamt eingeschaltet
        \item Unsere Tätigkeiten wurden nochmals erläutert
        \item November 2015: Landesamt hat den Bund eingeschaltet
        \item Immer mal wieder nach aktuellem Stand erkundigt
        \item 13.6.2016 Endlich das Ergebnis
    \end{itemize}
\end{frame}

\begin{frame}{Ergebnis Bundesebene}
    \begin{block}{..}
        \small
        Der Aufbau eines freien Kommunikationsnetzwerks und die Verwaltung von Servern,
        Richtfunkstrecken und Leistungen fallen aber bereits bisher nicht unter den Katalog der
        in §52 Abs. 2 Satz 1 AO genannten gemeinnützigen Zwecke. Die reine
        Zurverfügungstellung eines Netzzugangs ist nach AEAO zu §52 Nr.3 kein
        steuerbegünstigter Zweck. Durch den unentgeltlichen bzw. kostengünstigen Zugang zu
        Kommunikationsnetzwerken werden {\bf{}eigenwirtschaftliche Zwecke} der
        Zugangsberechtigten gefördert. Jeder Nutzer im Freifunk-Netz stellt seinen WLAN-
        Router für den Datentransfer {\bf{}der anderen Teilnehmer} zur Verfügung Im Gegenzug kann
        er aber Daten, wie z.B. Text, Musik und Filme über das interne Freifunk-Netz
        übertragen oder über von Teilnehmern eingerichtete Dienste im Netz chatten,
        telefonieren oder Online-Games spielen. Dies steht im Widerspruch zur
        vorgeschriebenen Selbstlosigkeit steuerbegünstigter Körperschaften (§55 Abs. 1 Nr. 1
        Satz 2 AO)
    \end{block}
\end{frame}

\begin{frame}{Gründung F3 Netze}
    \begin{itemize}
        \item 20.6.2016: Wir gründen trotzdem!
        \item Neues Finanzamt ist Schweinfurt
        \item 11.12.2016: Antrag auf Gemeinnützigkeit
        \item Es folgte eine informelle Ablehnung
        \item Wir haben wieder aufgeklärt
        \item Finanzamt: Bitte nehmt euren Antrag zurück
        \item Kein Rückzug, aber ruhen lassen
        \begin{itemize}
            \item Warten auf Politik
        \end{itemize}
    \end{itemize}
\end{frame}

\begin{frame}
    \begin{itemize}
        \item 33c3 (2016): Ludwig erklärt im offenen Vortrag den Klageweg
        \item Bundesratsinitiaitive läuft an
        \item Wir entschließen zu warten
        \item Ende 2017: Politik scheint aussichtslos
        \item Suche nach Partner für Spendenkampagne
        \item März 2018: Spendenkampange läuft an
        \item Juni 2018: RA-Franke nimmt unser Mandat auf
        \item August 2018: RA-Franke ist eingearbeitet
    \end{itemize}
\end{frame}

