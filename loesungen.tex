\section{Mögliche Lösungen}

\begin{frame}{Größten Baustellen}
    Meiner Meinung nach haben wir folgende größere (längerfristige)
    Baustellen:
    \begin{itemize}
        \item Netmon soll weg
        \item KeyXchange soll dezentralisiert werden
        \item Funk Verbindungen zwischen Hoods
    \end{itemize}
\end{frame}

\begin{frame}{Monitoring}
    \begin{itemize}
        \item Position und Owner Daten vom Gerät
        \item Webinterface nötig
    \end{itemize}
\end{frame}

\begin{frame}{Webinterface}
    \begin{itemize}
        \item \zb{} mit todo Tool von Bielefeld
        \item Koordinaten setzen
            \begin{itemize}
                \item wie offline setzen?
            \end{itemize}
        \item Kontakt-Daten
        \item Kein "GUI" Browser kann ordentlich mit IPv6 Link-Local
            umgehen
        \item Knoten müssen eine IP bekommen
        \item DHCP IP wäre immer anders, wie Router finden?
        \item Dynamic DNS: MAC.node.fff.community
            \begin{itemize}
                \item Muss schnell gehen -> DHCP setzt das Dynamic DNS
                \item Wie mit offline knoten verfahren?
            \end{itemize}
        \item Jeder Knoten hat die selbe IP
%            Wenn wir eh eine DHCP IPv4 Adresse bekommen, dann können
%            wir doch
%            eigentlich dazu immer die "Netzadresse + 127" als IP
%            hinzufügen und
%            sämtliche Kommunikation ins Mesh zu/von der IP weg filtern.
%            Dann hätten wir eine Local-Node Implementierung die immer
%            in der eigenen
%            Hood gilt
        \item IP im Monitoring nachschlagen
    \end{itemize}
\end{frame}

\begin{frame}{KeyXchange dezentralisieren}
    \begin{itemize}
        \item Der Knoten soll selber die richtige Hood finden
        \item Welcher VPN Server muss angewählt werden?
        \begin{itemize}
            \item VPN Server müssten alle Verbindungen annehmen (kein
                Problem mit fastd)
        \end{itemize}
        \item Welche Wifi Settings müssen gewählt werden? (wenn z.B.
            kein VPN da ist?)
        \item Zwei Dinge werden benötigt: 
        \begin{itemize}
            \item Hood-Configs enthalten VPN Server
            \item Scanner kann nach anderen Hoods (auch unbekannten)
                suchen, kompatibilität prüfen und sich einwählen
        \end{itemize}
    \end{itemize}
\end{frame}

\begin{frame}{Hood-Configs zum Gerät bringen}
    \begin{itemize}
        \item todo
    \end{itemize}
\end{frame}

\begin{frame}{Neue Hoods erzeugen}
    \begin{itemize}
        \item todo
    \end{itemize}
\end{frame}

\begin{frame}{L3 Richtfunk}
    \begin{itemize}
        \item todo
    \end{itemize}
\end{frame}

