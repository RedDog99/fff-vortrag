\section{Mögliche Lösungen}

\begin{frame}{Größten Baustellen}
    Größere (längerfristige) Baustellen:
    \begin{itemize}
        \item Netmon soll weg
        \item KeyXchange soll dezentralisiert werden
        \item Funk Verbindungen zwischen Hoods
    \end{itemize}
\end{frame}

\begin{frame}{Monitoring}
    \begin{itemize}
        \item Netmon abschalten, Abhängigkeiten lösen
        \begin{itemize}
            \item keyXchange
            \item neues Monitoring
        \end{itemize}
        \item Position und Owner Daten vom Gerät
        \item Webinterface nötig
    \end{itemize}
\end{frame}

\begin{frame}{Webinterface}
    \begin{itemize}
        \item \zb{} mit haserl (Freifunk Bielefeld)
        \item \zb{} mit lua (OpenWRT)
        \item Parameter setzen
        \begin{itemize}
            \item Koordinaten setzen
                \begin{itemize}
                    \item Lösung gesucht: Kartenansich ohne Netz?
                \end{itemize}
            \item Kontakt-Daten
        \end{itemize}
        \item Problem: ''GUI'' Browser und IPv6 Link-Local
    \end{itemize}
\end{frame}

\begin{frame}{IP für Knoten}
    \begin{itemize}
        \item Knoten müssen eine normale IP bekommen
        \item DHCP IP immer anders $\rightarrow$ wie Router finden?
        \begin{itemize}
            \item Dynamic DNS: MAC.node.fff.community
            \begin{itemize}
                \item Muss schnell gehen $\rightarrow$ DHCP setzt das Dynamic DNS
                \item Problem: offline Knoten
            \end{itemize}
            \item Local-Node IP:
            \begin{itemize}
                \item Jeder Knoten hat die selbe IP
                \item Beispiel: Vom DHCP zugewiesene Netzadresse +127
                \item Local-Node IP wird nicht ins Mesh geroutet
                \item Problem: Geht nur bei größeren Netzen
            \end{itemize}
            \item IP im Monitoring nachschlagen
            \begin{itemize}
                \item Problem: offline Knoten
            \end{itemize}
        \end{itemize}
    \end{itemize}
\end{frame}

\begin{frame}{KeyXchange}
    \begin{itemize}
        \item Dezentralisieren: Knoten soll selber seine Hood finden
        \begin{itemize}
            \item VPN Server auswählen
            \item VPN Server alle Verbindungen akzeptieren
        \end{itemize}
        \item Wifi Settings auswählen
        \begin{itemize}
            \item Problem: wenn z.B. kein VPN da ist?
        \end{itemize}
        \item Zwei Dinge werden benötigt: 
        \begin{itemize}
            \item Mögliche VPN Server
            \item Daten der Hood (Wifi / Routing / etc)
        \end{itemize}
    \end{itemize}
\end{frame}

\begin{frame}{Hood-Configs}
    \begin{columns}
    \column{0.45\textwidth}
        \begin{block}{Hood Config}
            \footnotesize
            [general]\\
            version=1\\[1ex]
            [hood]\\
            name=fuerth\\
            bssid=ca:ff:ee:ba:be:00\\
            protocol=batman-adv-v14\\
            channel2=1\\
            mode2=ht20+\\
            type=802.11s\\
            location=49.4814;10.966\\
            noAutoConnect=true\\[1ex]
            [network]\\
            subnet=10.16.44.0/27
        \end{block}
    \column{0.45\textwidth}
        \begin{block}{Gateway Config}
            \footnotesize
            [general]\\
            version=1\\[1ex]
            [network]\\
            hood=fuerth\\
            ip4range=10.50.44.1-.46.255\\
            //ip6range=2001:xx:xx:xx/64\\[1ex]
            [vpn]\\
            protocol=fastd\\
            address=vpn1.fff.community\\
            port=10000\\
            key=$<$fastd-public-key$>$\\[1ex]
            [sign]\\
            key=$<$public-key$>$\\
        \end{block}
    \end{columns}
\end{frame}

\begin{frame}{Hood-Configs verteilen}
    \begin{itemize}
        \item Gesucht: Synchronisation
        \item Überschreiben und Löschen muss abgesichert sein.
        \begin{itemize}
            \item[$\rightarrow$] Synchronisation darf nur für gültig signierte Dateien erfolgen
        \end{itemize}
    \end{itemize}
\end{frame}

\begin{frame}{L3 Richtfunk}
    \center
    \includegraphics[width=0.9\textwidth]{img/dia/l3richtfunk.pdf}
\end{frame}

