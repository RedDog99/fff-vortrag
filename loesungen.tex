\section{Mögliche Lösungen}

\begin{frame}{Größten Baustellen}
    Größere (längerfristige) Baustellen:
    \begin{itemize}
        \item Netmon soll weg
        \item KeyXchange soll dezentralisiert werden
        \item Funk Verbindungen zwischen Hoods
    \end{itemize}
\end{frame}

\begin{frame}{Monitoring}
    \begin{itemize}
        \item Netmon abschalten, Abhängigkeiten lösen
        \begin{itemize}
            \item keyXchange
            \item neues Monitoring
        \end{itemize}
        \item Position und Owner Daten vom Gerät
        \item Webinterface nötig
    \end{itemize}
\end{frame}

\begin{frame}{Webinterface}
    \begin{itemize}
        \item \zb{} mit haserl (Freifunk Bielefeld)
        \item \zb{} mit lua (OpenWRT)
        \item Parameter setzen
        \begin{itemize}
            \item Koordinaten setzen
                \begin{itemize}
                    \item Lösung gesucht: Kartenansich ohne Netz?
                \end{itemize}
            \item Kontakt-Daten
        \end{itemize}
        \item Problem: ''GUI'' Browser und IPv6 Link-Local
    \end{itemize}
\end{frame}

\begin{frame}{IP für Knoten}
    \begin{itemize}
        \item Knoten müssen eine normale IP bekommen
        \item DHCP IP immer anders $\rightarrow$ wie Router finden?
        \begin{itemize}
            \item Dynamic DNS: MAC.node.fff.community
            \begin{itemize}
                \item Muss schnell gehen $\rightarrow$ DHCP setzt das Dynamic DNS
                \item Problem: offline Knoten
            \end{itemize}
            \item Local-Node IP:
            \begin{itemize}
                \item Jeder Knoten hat die selbe IP
                \item Beispiel: Vom DHCP zugewiesene Netzadresse +127
                \item Local-Node IP wird nicht ins Mesh geroutet
                \item Problem: Geht nur bei größeren Netzen
            \end{itemize}
            \item IP im Monitoring nachschlagen
            \begin{itemize}
                \item Problem: offline Knoten
            \end{itemize}
        \end{itemize}
    \end{itemize}
\end{frame}

\begin{frame}{KeyXchange dezentralisieren}
    \begin{itemize}
        \item Der Knoten soll selber die richtige Hood finden
        \item Welcher VPN Server muss angewählt werden?
        \begin{itemize}
            \item VPN Server müssten alle Verbindungen annehmen (kein
                Problem mit fastd)
        \end{itemize}
        \item Welche Wifi Settings müssen gewählt werden? (wenn z.B.
            kein VPN da ist?)
        \item Zwei Dinge werden benötigt: 
        \begin{itemize}
            \item Hood-Configs enthalten VPN Server
            \item Scanner kann nach anderen Hoods (auch unbekannten)
                suchen, kompatibilität prüfen und sich einwählen
        \end{itemize}
    \end{itemize}
\end{frame}

\begin{frame}{Hood-Configs zum Gerät bringen}
    \begin{itemize}
        \item todo
    \end{itemize}
\end{frame}

\begin{frame}{Neue Hoods erzeugen}
    \begin{itemize}
        \item todo
    \end{itemize}
\end{frame}

\begin{frame}{L3 Richtfunk}
    \includegraphics[width=0.75\textwidth]{img/dia/l3richtfunk.pdf}
    \begin{itemize}
        \item todo
    \end{itemize}
\end{frame}

