\section{Stand von heute}

\begin{frame}{Freifunk Knoten}
    \begin{itemize}
        \item Kein Webinterface
        \item Nur SSH Zugang
        \item Hat nur eine Link-Local IPv6 Adresse
    \end{itemize}

    todo Hier muss noch mehr Inhalt her!
\end{frame}

\begin{frame}{Knoten VPN}
    \begin{itemize}
        \item Verwendetes VPN: fastd
        \begin{itemize}
            \item Multi-Client zu Multi-Client VPN
            \item Kein internes Routing
            \item Kein Forwarding
            \item Layer-II Netz
            \item Wir nutzen keine Verschlüsselung (! :-O)
        \end{itemize}
        \item Damit jeder jeden Erreichen kann nutzen wir Batman-Adv auf dem VPN
        \begin{itemize}
            \item Batman-Adv 2014 (!)
        \end{itemize}
    \end{itemize}
\end{frame}

\begin{frame}{Knoten VPN}
    \begin{itemize}
        \item Fastdstart.sh
        \begin{itemize}
            \item Legt config von fastd sowie up/down scripts an
            \item Startet fastd
            \item Meldet sich beim VPN-KeyXchange an
            \item Lädt Liste mit Peers
            \item Refresht fastd
        \end{itemize}
    \end{itemize}
\end{frame}

\begin{frame}{VPN-KeyXchange}
    \begin{itemize}
        \item Zentrale Webseite
        \item Knoten Identifizierung über MAC, alternativ über Name
        \item Aufteilung in ,,hood''s:
        \begin{itemize}
            \item Stellt ein Layer-II Netz dar
            \item Ein Gateway kann mehrere Layer-II Netze bedienen
        \end{itemize}
        \item Jeweils pro hood:
        \begin{itemize}
            \item Clients bekommen eine Liste aller Gateways
            \item Gateways bekommen eine Liste aller Clients+Gateways
        \end{itemize}
        \item Standort des Routers wird im Netmon anhand der MAC ermittelt
        \item Die Hood, welche am nächsten dran ist (voronoi) wird zugewiesen
        \item Problem: Abhängigkeit vom Netmon
    \end{itemize}
\end{frame}

\begin{frame}{VPN Server}
    In jeder Hood gibt es mehrere davon
    \begin{itemize}
        \item VPN Server
        \begin{itemize}
            \item Je Hood eine Instanz
        \end{itemize}
        \item Hoodzuweisung manuell im KeyXchange
    \end{itemize}
\end{frame}

\begin{frame}{Gateways}
    todo Läuft meist auf einem VPN Server

    todo Bis hierhin war alles Layer II.

    \begin{itemize}
        \item DHCP
        \item Aktuell ausschließlich IPv4
        \begin{itemize}
            \item Ungleiche Server Auslastung durch schlechtes DHCP Timing
            \item Batman-Adv GW Selection
        \end{itemize}
        \item DNS
        \item Policy base routing
        \item VPN (GRE) Tunnel zu anderen Gateways
        \item OLSR
        \begin{itemize}
            \item Routing zu anderen Gateways
        \end{itemize}
    \end{itemize}
\end{frame}

\begin{frame}{Internet Traffic}
    todo Aufteilung von Gateways in Hood-Server und Gateways.
    \begin{itemize}
        \item OLSR
        \item Dynamic Gateway Plugin
        \begin{itemize}
            \item Optional Routing ins Internet
            \item Problem: Ungleiche Server Auslastung durch gleichbeidene Routen-Qualität
        \end{itemize}
    \end{itemize}
\end{frame}

\begin{frame}{Netmon}
    \begin{itemize}
        \item Nodewatcher
        \begin{itemize}
            \item Generiert Status-Daten
            \item XML
            \item alle 5 Minuten
        \end{itemize}
        \item Configurator
        \begin{itemize}
            \item Verknüpft Netmon und Knoten
        \end{itemize}
        \item Crawler
        \begin{itemize}
            \item Sammelt Status-Daten
            \item Download über http Schnittstelle der Knoten
            \item Alles über Link-Local
        \end{itemize}
        \item Netmon
        \begin{itemize}
            \item Visualisiert Status-Daten
        \end{itemize}
    \end{itemize}
\end{frame}

\begin{frame}{Monitoring}
    \begin{itemize}
        \item Alfred
        \begin{itemize}
            \item XML
            \item alle 5 Minuten
            \item Multicast
        \end{itemize}
        \item Multicast Daten können auf mehreren Gateways gelesen werden
        \item Die Daten können von dort an eine zentrale Webseite (Karte, Etc) geschickt werden
        \item Standort und Ansprechpartner wird aus dem Netmon geladen
        \item Problem: Abhängig vom Netmon
    \end{itemize}
\end{frame}

\begin{frame}{Domain-Name-System}
    \begin{itemize}
        \item ganz Neu.. ! :-)
        \item Bisher: Nur für Namensauflösung ins Internet
        \item fff.community
        \item Mehrere DNS Server
        \item Synchronisation über dig Script
        \begin{itemize}
            \item todo Näher beschreiben
        \end{itemize}
        \item todo
    \end{itemize}
\end{frame}

\begin{frame}{Firmware Bau}
    \begin{itemize}
        \item Zentrales "files" Verzeichnis
        \item Pro System ein .config und mache überschreibende "files"
        \item Template System für Communities
    \end{itemize}
\end{frame}

\begin{frame}{Zusammenfasssung}
    \begin{itemize}
        \item Erfolgreich das L2 Netz in mehrere L3 Netze geteilt
        \item VPN Schlüsseltausch über zentrales Tool
        \item Zuordnung zu Hood über Standort aus Netmon
        \item todo
    \end{itemize}
\end{frame}
