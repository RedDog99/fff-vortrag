\section{Stand von heute}

\begin{frame}{Freifunk Knoten}
    \begin{itemize}
        \item Ziel: Ohne Konfiguration
        \begin{itemize}
            \item Keine DHCP Ranges
            \item LAN Ports vorkonfiguriert
            \item Ausnahme: Passwort für root Login
        \end{itemize}
        \item Hat nur eine Link-Local IPv6 Adresse
        \item Nur SSH Zugang
        \begin{itemize}
            \item[$\rightarrow$] Kein Webinterface
        \end{itemize}
        \item Batman-Adv 2013 (!)
    \end{itemize}
\end{frame}

\begin{frame}{Knoten VPN}
    \begin{itemize}
        \item Verwendetes VPN: fastd
        \item Wir nutzen keine Verschlüsselung (! :-O)
        \item Script zum Austausch der VPN Peers über den KeyXchange
    \end{itemize}
\end{frame}

\begin{frame}{VPN-KeyXchange}
    \begin{itemize}
        \item Zentrale Webseite
        \item Knoten meldet sich beim VPN-KeyXchange an
        \item Knoten lädt Liste mit Peers
        \item Knoten Identifizierung über MAC, alternativ über Name
        \item<2> Aufteilung in ,,hood''s:
        \begin{itemize}
            \item Die Hood, welche am nächsten dran ist (voronoi) wird zugewiesen
            \begin{itemize}
                \item Standort des Routers wird im Netmon anhand der MAC ermittelt
                \item Problem: Abhängigkeit vom Netmon
            \end{itemize}
            \item Clients bekommen eine Liste aller Server
            \item Server bekommen eine Liste aller Clients+Server
        \end{itemize}
        \item<2> Problem: Funk Verbindung zwischen den Hoods
    \end{itemize}
\end{frame}

\begin{frame}{VPN Server}
    \begin{itemize}
        \item VPN Server: In jeder Hood gibt es mehrere davon
        \item Hoodzuweisung manuell im KeyXchange
        \item DHCP
        \begin{itemize}
            \item Aktuell ausschließlich IPv4
            \item Ungleiche Server Auslastung durch schlechtes DHCP Timing
            \item Batman-Adv GW Selection
        \end{itemize}
        \item DNS Namesauflösung
        \item Policy base routing
        \item VPN (GRE) Tunnel zu anderen Gateways
        \item OLSR
        \begin{itemize}
            \item Routing zu anderen Gateways
        \end{itemize}
    \end{itemize}
\end{frame}

\begin{frame}{Gateways}
    \begin{itemize}
        \item Verbindet Freifunk und Internet
        \item IPv4 NAT (oft übers Ausland) ins Internet
        \item Announced 0.0.0.0/0 via OLSR
        \begin{itemize}
            \item Dynamic Gateway Plugin
            \item VPN Server können diese Routen nutzen
            \item Problem: Ungleiche Server Auslastung durch statische Routen-Qualität
        \end{itemize}
    \end{itemize}
\end{frame}

\begin{frame}{Netmon}
    \begin{itemize}
        \item Nodewatcher
        \begin{itemize}
            \item Generiert Status-Daten
            \item XML
            \item alle 5 Minuten
        \end{itemize}
        \item Configurator
        \begin{itemize}
            \item Verknüpft Netmon und Knoten
        \end{itemize}
        \item Crawler
        \begin{itemize}
            \item Sammelt Status-Daten
            \item Download über http Schnittstelle der Knoten
            \item Alles über Link-Local
            \begin{itemize}
                \item Muss in jeder Hood drin sein
            \end{itemize}
        \end{itemize}
        \item Netmon
        \begin{itemize}
            \item Visualisiert Status-Daten
            \item Mit den vielen Crawls hoffnungslos überfordert
        \end{itemize}
    \end{itemize}
\end{frame}

\begin{frame}{Monitoring}
    \begin{itemize}
        \item Alfred
        \begin{itemize}
            \item XML
            \item alle 5 Minuten
            \item Multicast
        \end{itemize}
        \item Multicast Daten können auf mehreren Gateways gelesen werden
        \item Die Daten können von dort an eine zentrale Webseite (Karte, Etc) geschickt werden
        \item Standort und Ansprechpartner wird aus dem Netmon geladen
        \item Problem: Abhängig vom Netmon
    \end{itemize}
\end{frame}

\begin{frame}{Domain-Name-System}
    \begin{itemize}
        \item ganz Neu.. ! :-)
        \item fff.community
        \begin{itemize}
            \item Langer Name, aber
            \item Keine Kollision
        \end{itemize}
        \item Mehrere DNS Server
        \item Zonen-Synchronisation über dig axfr
        \item Subdelegation an synchronisierte Hosts möglich
        \item Noch keine reverse delegation
    \end{itemize}
\end{frame}

\begin{frame}{Firmware Bau}
    \begin{itemize}
        \item Basiert auf OpenWRT
        \item "Eigenes" Framework (Buildscript)
        \item Zentrales "files" Verzeichnis
        \item Board-Support-Packages
        \begin{itemize}
            \item Ein .config
            \item Überschreibende "files"
        \end{itemize}
        \item Template System für versch. Communities
    \end{itemize}
\end{frame}

\begin{frame}{Zusammenfasssung}
    \begin{itemize}
        \item Erfolgreich das L2 Netz in mehrere L3 Netze geteilt
        \item VPN Schlüsseltausch über zentrales Tool
        \begin{itemize}
            \item Zuordnung zu Hood über Standort aus Netmon
            \item Manuelle Zuweisung der VPN Server
            \item Keine Anpassung an Firmware
            \begin{itemize}
                \item Fehlerhafte Verbindung (loop) der Hoods durch versehentliches Meshing
            \end{itemize}
        \end{itemize}
        \item Kein Traffic Balancing zwischen VPN Server und Gateways
        \item Netmon überlastet
        \item Netmon muss zu jeder Hood eine L2 Verbindung haben
        \item Neues Monitoring kennt Knoten Standort nur über Netmon
        \item Keine Möglichkeit die L2 Netze per Richtfunk zu verbinden
    \end{itemize}
\end{frame}
