\section{Stand von heute}

\begin{frame}{Fastdstart.sh}
    \begin{itemize}
        \item Legt config von fastd sowie up/down scripts an
        \item Startet fastd
        \item Meldet sich beim VPN-KeyXchange an
        \item Lädt Liste mit Peers
        \item Refresht fastd
    \end{itemize}
\end{frame}

\begin{frame}{VPN-KeyXchange}
    \begin{itemize}
        \item Knoten Identifizierung über MAC, alternativ über Name
        \item Aufteilung in ,,hood''s:
        \begin{itemize}
            \item Stellt ein Layer-II Netz dar
            \item Ein Gateway kann mehrere Layer-II Netze bedienen
        \end{itemize}
        \item Jeweils pro hood:
        \begin{itemize}
            \item Clients bekommen eine Liste aller Gateways
            \item Gateways bekommen eine Liste aller Clients+Gateways
        \end{itemize}
    \end{itemize}
\end{frame}

\begin{frame}{Internet Traffic}
    Aufteilung von Gateways in Hood-Server und Gateways.
\end{frame}

\begin{frame}{Netmon}
    \begin{itemize}
        \item Nodewatcher
        \begin{itemize}
            \item Generiert Status-Daten
            \item XML
            \item alle 5 Minuten
        \end{itemize}
        \item Configurator
        \begin{itemize}
            \item Verknüpft Netmon und Knoten
        \end{itemize}
        \item Crawler
        \begin{itemize}
            \item Sammelt Status-Daten
            \item Download über http Schnittstelle der Knoten
            \item Alles über Link-Local
        \end{itemize}
        \item Netmon
        \begin{itemize}
            \item Visualisiert Status-Daten
        \end{itemize}
    \end{itemize}
\end{frame}
