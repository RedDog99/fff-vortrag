\section{Freifunk V2}

\begin{frame}{}
    \begin{center}
        Freifunk V2
     \end{center}
\end{frame}

\begin{frame}{Die Idee}
    Neuer keyXchange ...

    \begin{itemize}
        \item KeyXchange stellt Hood-File bereit
        \item Knoten konfigurieren
        \begin{itemize}
            \item VPN
            \item WLAN
            \item IPv6 Netz
        \end{itemize}
        \item[$\rightarrow$] Unbeabsichtigte Hood-Verbindungen werden vermieden
    \end{itemize}
\end{frame}

\begin{frame}{Hood-File}
    \begin{itemize}
        \item Version
        \item ULA Prefix
        \item VPN Zugang: Protokoll, IP, Port, Key
        \item Hood-Daten: 
        \begin{itemize}
            \item Name, AP-ESSID
            \item BSSID, MESH\_ID, MESH\_ESSID
            \item Mesh-Protocol (Batman-Adv)
            \item 2.4 und 5 GHz Kanal, Bandbreite, Mesh\_Type
            \item Upgrade URL, Zeit-Server
            \item Zeit-Stempel
            \item Standort der Hood
        \end{itemize}
    \end{itemize}
\end{frame}

\begin{frame}{Der Uplink Knoten}
    keyXchange erreichbar

    Download Hood-File

    Konfigurieren

    Fertig.. Prima..
\end{frame}

\begin{frame}{Der WiFi Mesh Knoten}
    Kein keyXchange ..

    Keine Nachbarn (ich kenne die WLAN Config nicht)

    Woher Hood-File nehmen?

    Suchen?
\end{frame}

\begin{frame}{Der Ethernet Mesh Knoten}
    Kein keyXchange ..

    Batman-Adv läuft ja schon, Prima..

    WLAN-Settings? Woher nehmen?
    
    -> Hood-File vom Gateway laden

    Shit: Uplink-Knoten ändert sich

    Shit: Gateway und Uplink-Knoten sind unterschiedlich

    Shit: Es gibt mehrere Nachbarn -> Hood verbunden..

    Shit: Ich bin Ethernet Mesh Knoten und Uplink gleichzeitig -> ggfs Hood verbunden
\end{frame}

\begin{frame}{Config-AP am Uplink Knoten}
    Zusätzlicher (hidden) AP am Knoten

    Webserver

    Stellt bestätigtes Hood-File zum Download bereit
\end{frame}

\begin{frame}{Sector-File}
    Coole Sache

    Zweck .. usw
\end{frame}
