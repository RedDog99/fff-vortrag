\section{Freifunk V2}

\begin{frame}{}
    \begin{center}
        Freifunk V2
     \end{center}
\end{frame}

\begin{frame}{Die Idee}
    Neuer keyXchange ...

    \begin{itemize}
        \item Knoten schickt Standort
        \item KeyXchange stellt Hood-File (passend zum Standort) bereit
        \item Knoten konfiguriert
        \begin{itemize}
            \item VPN
            \item WLAN
            \item IPv6 Netz
        \end{itemize}
        \item[$\rightarrow$] Unbeabsichtigte Hood-Verbindungen werden vermieden
    \end{itemize}
\end{frame}

\begin{frame}{Hood-File}
    \begin{itemize}
        \item Version
        \item ULA Prefix
        \item VPN Zugang: Protokoll, IP, Port, Key
        \item Hood-Daten: 
        \begin{itemize}
            \item Name, AP-ESSID
            \item BSSID, MESH\_ID, MESH\_ESSID
            \item Mesh-Protokoll (Batman-Adv)
            \item 2.4 und 5 GHz Kanal, Bandbreite, Mesh\_Type
            \item Upgrade URL, Zeit-Server
            \item Zeit-Stempel
            \item Standort der Hood
        \end{itemize}
    \end{itemize}
\end{frame}

\begin{frame}{Freifunk V2}
    \center
    \only<1>{Klingt ja eigentlich ganz gut}
    \only<2>{... aber}
\end{frame}

\begin{frame}{Der Uplink Knoten}
    \begin{itemize}
        \item keyXchange erreichbar
        \item Download Hood-File
        \item Konfigurieren
        \item Fertig.. Prima..
    \end{itemize}
\end{frame}

\begin{frame}{Mesh Knoten mit WLAN}
    \begin{itemize}
        \item Kein WAN: kein keyXchange erreichbar
        \item Keine Nachbarn (Knoten kennt die WLAN/Mesh Config nicht)
        \item Woher Hood-File nehmen?
        \item Suchen?
        \item[$\rightarrow$] Config-AP
    \end{itemize}
\end{frame}

\begin{frame}{Config-AP}
    \begin{itemize}
        \item Zusätzlicher (hidden) AP am Knoten
        \item Kleiner Webserver
        \item Stellt Hood-File zum Download bereit
        \item Mesh Knoten mit WLAN können Hood-File downloaden
    \end{itemize}
\end{frame}

\begin{frame}{Mesh Knoten mit Ethernet}
    \begin{itemize}
        \item Kein WAN: kein keyXchange erreichbar
        \item Batman-Adv läuft ja schon, Prima...
        \item WLAN-Settings? Woher nehmen?
        \item[$\rightarrow$] Hood-File vom Gateway laden
        \item[:(] Uplink-Knoten ändert sich
        \item[:(] Gateway und Uplink-Knoten sind unterschiedlich
        \item[:(] Es gibt mehrere Nachbarn: $\rightarrow$ Hood verbunden..
        \item[:(] Mesh Knoten bekommt Uplink: $\rightarrow$ ggfs Hood verbunden
    \end{itemize}
\end{frame}

\begin{frame}{Konstellationen}
    \begin{itemize}
        \item Knoten per Ethernet an Knoten per Ethernet an Uplink Knoten
        \item Knoten per Ethernet an Knoten per WLAN an Uplink Knoten
        \item Knoten per WLAN an Knoten per Ethernet an Uplink Knoten
        \item Knoten per WLAN an Knoten per WLAN an Uplink Knoten
        \item Mehrere Uplink Knoten
        \item Knoten per Ethernet und WLAN an Uplink
        \item Knoten mit 5 GHz und 2.4 GHz
        \item ... viele mehr ...
    \end{itemize}
\end{frame}

\begin{frame}{Sector-File}
    \begin{itemize}
        \item Ein Sector-File kann WLAN-Mesh Parameter überschreiben:
        \begin{itemize}
            \item WLAN Kanal
            \item Typ
            \item Mesh-ID
            \item AP-ESSID
        \end{itemize}
        \item Sector-File wird über Config-AP verteilt
        \begin{itemize}
            \item[:(] Nicht auf Ethernet
            \item[$\rightarrow$] Noch mehr Konstellationen
        \end{itemize}
    \end{itemize}
    \vfill
    \raggedleft
    \only<2>{... Ich erspare euch das weitere}
\end{frame}
