\section{Freifunk V2}

\begin{frame}{}
    \begin{center}
        Freifunk V2
     \end{center}
\end{frame}

\begin{frame}{Die Idee}
    Neuer keyXchange ...

    \begin{itemize}
        \item Knoten schickt Standort
        \item KeyXchange stellt Hood-File (passend zum Standort) bereit
        \item Knoten konfiguriert
        \begin{itemize}
            \item VPN
            \item WLAN
            \item IPv6 Netz
        \end{itemize}
        \item[$\rightarrow$] Unbeabsichtigte Hood-Verbindungen werden vermieden
    \end{itemize}
\end{frame}

\begin{frame}{Hood-File}
    \begin{itemize}
        \item Version
        \item ULA Prefix
        \item VPN Zugang: Protokoll, IP, Port, Key
        \item Hood-Daten: 
        \begin{itemize}
            \item Name, AP-ESSID
            \item BSSID, MESH\_ID, MESH\_ESSID
            \item Mesh-Protokoll (Batman-Adv)
            \item 2.4 und 5 GHz Kanal, Bandbreite, Mesh\_Type
            \item Upgrade URL, Zeit-Server
            \item Zeit-Stempel
            \item Standort der Hood
        \end{itemize}
    \end{itemize}
\end{frame}

\begin{frame}{Freifunk V2}
    \center
    \only<1>{Klingt ja eigentlich ganz gut}
    \only<2>{... aber}
\end{frame}

\begin{frame}{Der Uplink Knoten}
    \begin{itemize}
        \item keyXchange erreichbar
        \item Download Hood-File
        \item WLAN/Mesh/etc Konfigurieren
        \item Fertig.. Prima..
    \end{itemize}
\end{frame}

\begin{frame}{Mesh Knoten mit WLAN}
    \begin{itemize}
        \item Kein WAN: kein keyXchange erreichbar
        \item Noch keine Nachbarn (Knoten kennt die WLAN/Mesh Config nicht):
        \begin{itemize}
            \item Woher Hood-File nehmen? Suchen?
            \item[$\rightarrow$] Config-AP
        \end{itemize}
        \item Hat schon Nachbarn (Knoten kennt die WLAN/Mesh Config):
        \begin{itemize}
            \item[$\rightarrow$] Hood-File vom Gateway laden
        \end{itemize}
        \item Nachbar-Knoten ändert sich (Keine WLAN/Mesh Verbindung mehr):
        \begin{itemize}
            \item[$\rightarrow$] Config-AP
        \end{itemize}
        \item Mesh Knoten bekommt Uplink und Hood ändert sich:
        \begin{itemize}
            \item[$\rightarrow$] Neue Mesh-Parameter
        \end{itemize}
        \item[:)] Alle Nachbarn haben selbe WLAN/Mesh config: Hoods bleiben getrennt
        \item[:(] Hood-File auf Gateway, keyXchangeV2 und ConfigAP können divergieren
    \end{itemize}
\end{frame}

\begin{frame}{Config-AP}
    \begin{itemize}
        \item Zusätzlicher (hidden) AP am Knoten
        \item Kleiner Webserver
        \item Stellt Hood-File zum Download bereit
        \item Mesh Knoten mit WLAN können Hood-File downloaden
    \end{itemize}
\end{frame}

\begin{frame}{Mesh Knoten mit Ethernet}
    1. Versuch
    \begin{itemize}
        \item Kein WAN: kein keyXchange erreichbar
        \item Batman-Adv läuft ja schon, Prima...
        \item WLAN-Settings? Woher nehmen?
        \item[$\rightarrow$] Hood-File vom Gateway laden
        \item[:(] Uplink-Knoten ändert sich
        \item[:(] Es gibt mehrere Nachbarn: $\rightarrow$ Hood verbunden..
        \item[:(] Mesh Knoten bekommt Uplink: $\rightarrow$ ggfs Hood verbunden
    \end{itemize}
\end{frame}

\begin{frame}{Mesh Knoten mit Ethernet}
    2. Versuch: MacKnocker
    \begin{itemize}
        \item Kein WAN: kein keyXchange erreichbar
        \item MacKnocker blockiert alle Batman-Adv Verbindungen
        \begin{itemize}
            \item[$\rightarrow$] IPv6 LL funktioniert auf dem Link trotzdem
        \end{itemize}
        \item WLAN-Settings? Woher nehmen?
        \begin{itemize}
            \item Alle Nachbarn suchen (Multicast Ping)
            \item Hood-File vom Nachbarn downloaden (Subset vom Config-AP)
            \item WLAN/Mesh/etc Konfigurieren + MacKnocker mit Hood-ID restarten
            \item[$\rightarrow$] MacKnocker gibt Batman-Adv für andere MacKnocker der selben Hood frei
        \end{itemize}

        \item[:)] Alle Nachbarn haben selbe MacKnocker config: Hoods bleiben getrennt
        \item[:(] Hood-File auf Gateway, keyXchangeV2 und ConfigAP können divergieren
    \end{itemize}
\end{frame}

\begin{frame}{Sector-File}
    \only<2>{
        \begin{textblock*}{\textwidth}(0.05\textwidth,0.3\textheight)
            \includegraphics[width=\textwidth,height=0.5\textheight]{img/strike}
        \end{textblock*}
    }
    \begin{itemize}
        \item Ein Sector-File kann WLAN-Mesh Parameter überschreiben:
        \begin{itemize}
            \item WLAN Kanal
            \item Typ
            \item Mesh-ID
            \item AP-ESSID
        \end{itemize}
        \item Sector-File wird über Config-AP verteilt
        \begin{itemize}
            \item[:(] Nicht auf Ethernet
            \item[$\rightarrow$] Noch mehr Konstellationen
        \end{itemize}
    \end{itemize}
    \begin{textblock*}{0.3\textwidth}(0.65\textwidth,0.5\textheight)
        \includegraphics[width=\textwidth]{img/dia/sector}
    \end{textblock*}
    %\only<2>{
    %}
\end{frame}

\begin{frame}{Zusammenfassend}
    \begin{itemize}
        \item Wir haben es geschafft! Juhuu!
        \item Krass geile Technik
        \item Mega kompliziert
        \begin{itemize}
            \item Auch das Gateway ist komplexer
        \end{itemize}
        \item Entwickler haben ewig gebraucht
        \begin{itemize}
            \item Weiterentwicklung unklar (viele sind nach Fertigstellung abgesprungen)
        \end{itemize}
        \item Hotspot-Betreiber unzufrieden mit neuen ESSID's
        \item Ziele von Freifunk bleiben auf der Strecke
        \item ...
    \end{itemize}
\end{frame}
