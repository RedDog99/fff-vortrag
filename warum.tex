\section*{Warum Gemeinnützigkeit?}
\begin{frame}{Warum Gemeinnützigkeit}
    \begin{center}
        Warum Gemeinnützigkeit
    \end{center}
\end{frame}

\begin{frame}{Warum Gemeinnützigkeit}
    \begin{itemize}
        \item Geld
        \begin{itemize}
            \item Spenden sind in der Einkommensteuer absetzbar
            \item Firmen können Spenden als Ausgabe absetzen
            \begin{itemize}
                \item[$\rightarrow$] Bessere Spendenbereitschaft
            \end{itemize}
            \item Beispiele:
            \begin{itemize}
                \item Geldspenden für \#18fff
                \item Hosting gegen Spendenbescheinigung
                \item Transit gegen Spendenbescheinigung
            \end{itemize}
            \item Steuerliche Vorteile innerhalb des Vereins
        \end{itemize}
        \item Anerkennung
        \begin{itemize}
            \item Bei Städten / Gemeinden
            \item Kooperationen
            \item Banken
        \end{itemize}
    \end{itemize}
\end{frame}

%\begin{frame}{AO §52 - Selbstlose Förderung}
%    \begin{block}{Absatz 1}
%        Eine Körperschaft verfolgt gemeinnützige Zwecke, wenn ihre Tätigkeit
%        darauf gerichtet ist, die {\bf{}Allgemeinheit auf materiellem, geistigem oder
%        sittlichem Gebiet selbstlos zu fördern}. Eine Förderung der Allgemeinheit
%        ist nicht gegeben, wenn der Kreis der Personen, dem die Förderung zugute
%        kommt, fest abgeschlossen ist, zum Beispiel Zugehörigkeit zu einer
%        Familie oder zur Belegschaft eines Unternehmens, oder infolge seiner
%        Abgrenzung, insbesondere nach räumlichen oder beruflichen Merkmalen,
%        dauernd nur klein sein kann. Eine Förderung der Allgemeinheit liegt
%        nicht allein deswegen vor, weil eine Körperschaft ihre Mittel einer
%        Körperschaft des öffentlichen Rechts zuführt.
%    \end{block}
%\end{frame}
%
%\begin{frame}{AO §52 - Katalogzwecke/Öffnung}
%    \begin{block}{Absatz 2}
%        Unter den Voraussetzungen des Absatzes 1 sind als Förderung der Allgemeinheit anzuerkennen:
%        \begin{enumerate}
%            \item die Förderung von Wissenschaft und Forschung;
%            \item die Förderung der Religion;
%            \item ..
%        \end{enumerate}
%        Sofern der von der Körperschaft verfolgte Zweck {\bf{}nicht unter Satz 1
%        fällt, aber die Allgemeinheit auf materiellem, geistigem oder sittlichem
%        Gebiet entsprechend selbstlos gefördert wird, kann dieser Zweck für
%        gemeinnützig erklärt werden}. Die obersten Finanzbehörden der Länder haben
%        jeweils eine Finanzbehörde im Sinne des Finanzverwaltungsgesetzes zu
%        bestimmen, die für Entscheidungen nach Satz 2 zuständig ist.
%    \end{block}
%\end{frame}
%
%\begin{frame}{AEAO §43 3.}
%    \begin{block}{Internetvereine}
%        Internetvereine können wegen Förderung der Volksbildung als gemeinnützig
%        anerkannt werden, sofern ihr Zweck nicht der Förderung der (privat
%        betriebenen) Datenkommunikation durch Zurverfügungstellung von Zugängen
%        zu Kommunikationsnetzwerken sowie durch den Aufbau, die Förderung und
%        den Unterhalt entsprechender Netze zur privaten und geschäftlichen
%        Nutzung durch die Mitglieder oder andere Personen dient.
%    \end{block}
%\end{frame}
